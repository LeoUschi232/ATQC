\begin{frame}{Measurement}{Main Principles}
    We want to measure observable \(g\in\mathcal{P}_n\)
    of state \(\ket{\psi}\),
    stabilized by \(\left\langle g_i\mid i\in\natur,1\leq i\leq l\right\rangle\). \\

    \onslide<2->{
        \vspace*{2mm}
        \textbf{Two possibilities:}
        \vspace*{1mm}
    }
    \begin{enumerate}
        \setlength{\itemsep}{0.25\baselineskip}
        \onslide<2->{
            \item
            \(g\) commutes with all generators of the stabilizer. \\
        }
        \onslide<3->{
            \(\qquad\Longrightarrow\)
            Measurement outcome is deterministic.
        }
        \onslide<4->{
            \item
            \(g\) anti-commutes with at least 1 generator of the stabilizer. \\
        }
        \onslide<5->{
            \(\qquad\Longrightarrow\)
            Measurement outcome is not deterministic.
        }
    \end{enumerate}

    \vspace*{25mm}

    \fullfootcite{01_QuantumComputationAndQuantumInformation}
\end{frame}

\begin{frame}{Measurement}{Deterministic case}
    \(g\) commutes with all \(g_i\) and assume \(g\) does not have a global phase.

    \onslide<2->{
        \[
            \forall i\holds
            g_ig\ket{\psi}
            =gg_i\ket{\psi}
            =g\ket{\psi}
            \means
            g\ket{\psi}\in V_S
        \]
    }
    \onslide<3->{
        \[
            g^2\ket{\psi}
            =I\ket{\psi}
            =\ket{\psi}
            \means
            g\ket{\psi}
            =\pm\ket{\psi}
            \means
            g\in S\veebar (-g)\in S
        \]
    }
    \[
        \begin{aligned}
            \onslide<4->{
                g&\in S
                &&\means
                g\ket{\psi}
                =\ket{\psi}
                &&\means
                \text{Measurement yields }+1 \\
            }
            \onslide<5->{
                (-g)&\in S
                &&\means
                g\ket{\psi}
                =-\ket{\psi}
                &&\means
                \text{Measurement yields }-1
            }
        \end{aligned}
    \]

    \vspace*{2mm}
    \onslide<6->{
        In both cases the measurement does not disturb the state of the system,
        and leaves the stabilizer invariant.
    }

    \vspace*{5mm}

    \fullfootcite{01_QuantumComputationAndQuantumInformation}
\end{frame}

\begin{frame}{Measurement}{Non-deterministic case preliminaries}
    Without loss of generality, let \(g\) anti-commute with \(g_1\) and \(g\) does not have a global phase.
    \vspace*{1mm}
    \[
        \begin{aligned}
            \onslide<2->{
                &\forall g_j\with j\neq1\aand g_jg=-gg_j:
                \text{ Replace }g_j\with g_j'=g_1g_j \\
            }
            \onslide<3->{
                &\qquad\Longrightarrow g_j'g=g_1g_jg=-g_1gg_j=gg_1g_j=gg_j' \\
            }
            \onslide<4->{
                &\qquad\qquad\Longrightarrow g\text{ commutes with }g_j'
            }
        \end{aligned}
    \]
    \onslide<5->{
        \(\Longrightarrow\)
        \(g\) only commutes with \(g_1\). \\
    }
    \vspace*{1mm}
    \onslide<6->{
        Because \(g\) has eigenvalues \(\pm1\),
        the measurement operators are: \(M_{\pm g}=\frac{I\pm g}{2}\)
    }

    \vspace*{15mm}

    \fullfootcite{01_QuantumComputationAndQuantumInformation}
\end{frame}

\begin{frame}{Measurement}{Non-deterministic case continuation}
    Measurement probabilities:
    \[
        p(+1)
        =\text{tr}\left(
        \frac{I+g}{2}\ket{\psi}\bra{\psi}
        \right)
        \tacticalAnd
        p(-1)
        =\text{tr}\left(
        \frac{I-g}{2}\ket{\psi}\bra{\psi}
        \right)
    \]
    \[
        \begin{aligned}
            \onslide<2->{
                p(+1)
                &=\text{tr}\left(
                \frac{I+g}{2}\ket{\psi}\bra{\psi}
                \right)
            }
            \onslide<3->{
                =\text{tr}\left(
                \frac{I+g}{2}g_1\ket{\psi}\bra{\psi}
                \right)
            }
            \onslide<4->{
                =\text{tr}\left(
                g_1\frac{I-g}{2}\ket{\psi}\bra{\psi}
                \right) \\
            }
            \onslide<5->{
                &=\text{tr}\left(
                \frac{I-g}{2}\ket{\psi}\bra{\psi}g_1
                \right)
            }
            \onslide<6->{
                =\text{tr}\left(
                \frac{I-g}{2}\ket{\psi}\bra{\psi}g_1^{\dagger}
                \right)
            }
            \onslide<7->{
                =\text{tr}\left(
                \frac{I-g}{2}\ket{\psi}\bra{\psi}
                \right)
            }
            \onslide<8->{
                =p(-1)
            }
        \end{aligned}
    \]
    \onslide<9->{
        \[
            p(+1)=p(-1)
            \aand
            p(+1)+p(-1)=1
            \means
            p(+1)=p(-1)
            =\frac{1}{2}
        \]
    }

    \vspace*{5mm}

    \fullfootcite{01_QuantumComputationAndQuantumInformation}
\end{frame}